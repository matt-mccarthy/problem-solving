\documentclass[notitlepage]{simple}

\def\ord{\operatorname{ord}}

\title{Weil Height and the $p$-adic Numbers}
\date{January 2016}
\author{Matt McCarthy\\\href{mailto:matthew.mccarthy.12@cnu.edu}{matthew.mccarthy.12@cnu.edu}}

\addbibresource{p-adic.bib}

\begin{document}

\maketitle

\section{Introduction}

Height functions generally measure the complexity of a number.
In particular, the Weil height measure the complexity of a rational number as follows.

\begin{definition}\label{weil1}
	Let $x=m/n\in\QQ$, then the \textit{Weil height of $x$}, denoted $h(x)$, is defined as
	\[
		h(x)=\max\set{|m|,|n|}.
	\]
\end{definition}

As nice as this height function is, we would like to have a natural extension to other spaces, like the algebraics, where measuring the Euclidean distance of the numerators and denominators makes less sense.

\section{Background}

Before we extend the Weil height, we introduce the concept of the $p$-adic ordinal and norm.

\begin{definition}[$p$-adic Ordinal \cite{koblitz}]
	Let $p$ be a prime and let $a\in\ZZ$.
	Then the \textit{$p$-adic ordinal} of $a$, denoted $\ord_p a$, is defined as
	\[
		\ord_p a = \max\set{n\text{ s.t. } p^n | a}.
	\]
	Furthermore, for any $x=b/c\in\QQ$,
	\[
		\ord_p x = \ord_p a - \ord_p b.
	\]
\end{definition}

Using the definition of $p$-adic ordinal, we now provide the definition of the $p$-adic norm.

\begin{definition}[$p$-adic Norm \cite{koblitz}]
	Let $p$ be a prime and $x\in\QQ$.
	Then the \textit{$p$-adic norm of $x$}, denoted $|x|_p$, is defined as
	\[
		|x|_p =
		\begin{cases}
			p^{-\ord_p x} & x\neq 0\\
			0 & x=0.
		\end{cases}
	\]
\end{definition}

\section{Problem}

We want to show that following redefinition of the Weil height is equivalent to \autoref{weil1}

\begin{thm}
	Let $\alpha=n/d\in\QQ$ with $n\neq 0$ and $\gcd(n,d)=1$.
	Define $h(\alpha)=\paren{\prod_p \max\set{1,|\alpha|_p}}\paren{\max\set{1,|\alpha|}}$.
	Then $h(\alpha)=\max\set{|n|,|d|}$.
\end{thm}
\begin{proof}
	Define $r(\alpha):=\prod_p \max\set{1,|\alpha|_p}$, then $h(\alpha)=r(\alpha)\cdot\paren{\max\set{1,|\alpha|}}$.
	Since $|n|,|d|\in\ZZ^+$, $|n|=\prod_{i=1}^k p_i^{n_i}$ and $|d|=\prod_{i=1}^j q_i^{m_i}$ where $\set{p_i}$ and $\set{q_i}$ are finite sets of prime numbers and $\set{n_i},\set{m_i}\subset\NN$.
	Furthermore, $\set{p_i}\cap\set{q_i}=\emptyset$ since $\gcd(n,d)=1$.
	Therefore,
	\begin{align*}
		r(\alpha)
		&=r\paren{\paren{\prod_{p_i} p_i^{n_i}}\paren{\prod_{q_i} q_i^{-m_i}} }\\
		&=\paren{\prod_{p_i} \max\set{1,|\alpha|_{p_i}}}\paren{\prod_{q_i} \max\set{1,|\alpha|_{q_i}}}\\
		&=\paren{\prod_{p_i} \max\set{1,p_i^{-n_i}}}\paren{\prod_{q_i} \max\set{1,q_i^{m_i}}}\\
		&=1\cdot\prod_{q_i} q_i^{m_i}\\
		&=|d|.
	\end{align*}
	Thus,
	\[
		h(\alpha)=|d|\cdot\max\set{1,|n/d|}.
	\]
	If $n<d$,
	\[
		h(\alpha) = |d|\cdot 1 = \max\set{|n|,|d|}.
	\]
	If $n=d$,
	\[
		h(\alpha) = |d| =\max\set{|n|,|d|}.
	\]
	Otherwise, $n> d$ and
	\[
		h(\alpha) = |d|\frac{|n|}{|d|}=|n|=\max{|n|,|d|}.
	\]
\end{proof}
\begin{analysis}
	We begin by splitting up $h$ into the product of $p$-adic norms and the Euclidean norm.
	From there, we notice that the product of $p$-adic norms is equal to $|d|$.
	To prove this, we use the fact that the prime factorizations of $n$ and $d$ are finite and disjoint, we can break $r(\alpha)$ into the product of $r(n)$ and $r(1/d)$.
	When we consider $r(n)$ we see that $|n|_p\leq 1$ for all primes $p$ and thus $\max\set{1,|n|_p}=1$.
	When we consider $r(1/d)$, we see that $|1/d|_p\geq 1$ for all $p$ and thus, $\max\set{1,|1/d|_p}=|1/d|_p$.
	For any prime not in $d$'s prime factorization, $|1/d|_p=1$, for those in $d$'s prime factorization, $|1/d|_p=p^{n_p}=\ord_p d$.
	Thus, $r(1/d)=|d|$.
	We then consider cases to get the result.
\end{analysis}

\printbibliography

\end{document}
