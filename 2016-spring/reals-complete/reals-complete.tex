\documentclass[notitlepage]{simple}

\author{Matt McCarthy}
\title{Completeness of $\RR$}
\date{Februrary 2016}

\begin{document}

\maketitle

\begin{thm*}
	The real numbers form a complete metric space.
\end{thm*}

\noindent Notation: We denote the set of non-negative real numbers as $\RR^+$.

We begin by providing a way to measure distances in a space.

\begin{definition}[Metric]
	Let $X$ be a set and $d:X\times X\rightarrow \RR^+$ be a map.
	We say $d$ is a \textit{metric} if and only if all of the following hold.
	\begin{enumerate}
		\item For all $x,y\in X$, $d(x,y)=0$ if and only if $x=y$.
		\item For all $x,y\in X$, $d(x,y)=d(y,x)$ (symmetric property).
		\item For all $x,y,z\in X$, $d(x,z)\leq d(x,y)+d(y,x)$ (triangle inequality).
	\end{enumerate}
	If $d$ is a metric on $X$, we say $(X,d)$ forms a \textit{metric space}.
\end{definition}

For example, $(\QQ,d)$ and $(\RR,d)$ where $d(x,y)=\abs{x-y}$ is a metric space,
and more generally, $(\RR^n, d_n)$ where $d_n$ is the Euclidean distance is also a metric space.
Now that we have a way to talk about distances in spaces we can talk about convergence of sequences in those spaces.

\begin{definition}[Convergent Sequence]
	Let $(X,d)$ be a metric space and let $\set{a_n}_{n\in\NN}$ be a sequence in $X$.
	We say $\set{a_n}_{n\in\NN}$ \textit{converges} to some $a\in X$, denoted $a_n\rightarrow a$ if and only if for all $\varepsilon>0$ there exists a $N_\varepsilon\in\NN$ such that $d(a_n,a)<\varepsilon$ for all $n\geq N_\varepsilon$.
\end{definition}

However, convergence is often too strong a condition to prove, and moreover convergence also requires a proposed limit.
Therefore, we introduce the notion of a \textit{Cauchy Sequence} where the terms get arbitrarily close to each other.

\begin{definition}[Cauchy Sequence]
	Let $(X,d)$ be a metric space and let $\set{a_n}_{n\in\NN}$ be a sequence in $X$.
	We say $\set{a_n}_{n\in\NN}$ is a \textit{Cauchy sequence} if and only if for any $\varepsilon>0$, there exists a $N_\varepsilon\in\NN$ such that $d(a_n,a_m)<\varepsilon$ for all $n,m\geq N_\varepsilon$.
\end{definition}

Note, however that Cauchy sequences are not always convergent in their space.
For example consider the sequence given by
\[
	a_n = \paren{1+\frac{1}{n}}^n.
\]
We know these terms are rational and thus $a_n$ is a $\QQ$-valued sequence.
Furthermore, the sequence is in fact Cauchy.
However, the limit of this sequence as $n$ approaches infinity is $e$ and thus, this Cauchy sequence does not converge in $\QQ$, but it does in $\RR$.

This leads us to a notion of \textit{completeness} in a metric space.

\begin{definition}[Complete Metric Space]
	Let $(X,d)$ be a metric space.
	Then we say $(X,d)$ is \textit{complete} if and only if every Cauchy sequence in $X$ converges in $X$.
\end{definition}
By this definition of completeness, $\QQ$ is not complete as shown by our prior example.
However, $\RR$ is complete.

Before we get into any theorems here, we need a notion of \textit{boundedness} for subsets of the real numbers.

\begin{definition}[Bounded]
	Let $S$ be a subset of the real numbers.
	Then $S$ is \textit{bounded} if and only if there exists an $M>0$ such that $S\subseteq [-M,M]$.
\end{definition}

We also need to invoke a famous theorem due to Bolzano and Weierstrass, which states that any bounded sequence in $\RR$ has a subsequence that is convergent in $\RR$.

\begin{thm}[Bolzano-Weierstrass]
	Let $\set{a_n}_{n\in\NN}$ be a bounded $\RR$-valued sequence.
	Then there exists a $\set{a_{n_k}}\subseteq\set{a_n}$ such that $a_{n_k}$ converges to some $a\in\RR$.
\end{thm}
\begin{proof}
	Let $\set{x_n}_{n\in\NN}$ be bounded.
	Then there exists an $M\in\NN$ such that $\set{x_n}_{n\in\NN}\subset[-M,M]\subset\RR$.
	Bisect $[-M,M]$ into $[-M,0],[0,M]$.
	At least one half has infinitely many sequence points, call this one $I_1$.
	Pick a sequence point $x_{n_1}\in I_1$.
	Bisect $I_1$ and call the half with infinitely many sequence points $I_2$.
	Pick a sequence point $x_{n_2}\in I_2$ such that the index $n_2>n_1$.
	Iterate this process, that is bisect $I_k$ and call the half with infinitely many points $I_{k+1}$.
	Then choose a sequence point $x_{n_{k+1}}\in I_{k+1}$ with the property that the index $n_{k+1}>n_k>\ldots > n_1$.
	Moreover, by nested interval property, $\bigcap_{k\in\NN} I_k\neq\emptyset$.
	Thus there exists at least one $x\in \bigcap_{k\in\NN} I_k$.
	We want to show that $x_{n_k}\rightarrow x$.
	Let $\varepsilon> 0$ be given.
	Choose a $K_\varepsilon\in\NN$ such that $2^{-K_\varepsilon}M< \varepsilon$.
	Note that for all $k\geq K_\varepsilon$, $2^{-k} M <  2^{-K_\varepsilon}M< \varepsilon$.
	Since $x\in \bigcap_{k\in\NN} I_k$, $x\in I_k$ for all $k\in\NN$.
	Thus for all $k\in\NN$, $|x_{n_k}-x|\leq 2^{-k}M$.
	Therefore, when we force $k\geq K_\varepsilon$ we have $I_k\subset I_{K_\varepsilon}$ and
	\[
	|x_{n_k}-x|\leq 2^{-k}M \leq 2^{-K_\varepsilon} M < \varepsilon.
	\]
	Thus, $\set{x_{n_k}}_{k\in\NN}$ is a subsequence of $\set{x_n}_{n\in\NN}$ that is convergent in $\RR$.
\end{proof}

Furthermore, we need the following lemma.

\begin{lemma}
	Any $\RR$-valued Cauchy sequence is bounded.
\end{lemma}
\begin{proof}
	Let $\varepsilon >0$ be given and let $\set{a_n}_{n\in\NN}$ be a Cauchy sequence in $\RR$.
	Therefore, there exists an $N_\varepsilon$ such that $|a_n-a_m|<\varepsilon$ for all $n,m\geq N_\varepsilon$.
	Therefore, for all $n\geq N_\varepsilon$, $|a_{N_\varepsilon}-a_n|<\varepsilon$.
	Thus for each $n\geq N_\varepsilon$, $a_n\in[a_{N_\varepsilon}-\varepsilon,a_{N_\varepsilon}+\varepsilon]$.
	Take $M_1=\max\set{a_{N_\varepsilon}-\varepsilon,a_{N_\varepsilon}+\varepsilon}$, which yields $\set{a_n}_{n=N_\varepsilon}^\infty\subset[-M_1,M_1]$.

	Furthermore, since $\set{a_n}_{n=1}^{N_\varepsilon-1}$ is finite, take $M_2=\max\set{|a_n|}_{n=1}^{N_\varepsilon-1}$.
	Thus, $\set{a_n}_{n=1}^{N_\varepsilon-1}\subset [-M_2,M_2]$.
	If we take $M=\max\set{M_1,M_2}$, then we have $\set{a_n}_{n\in\NN}\subset[-M,M]$.
	Therefore, the sequence is bounded.
\end{proof}

And now we provide the proof.

\begin{thm}
	The real numbers form a complete metric space.
\end{thm}
\begin{proof}
	Let $\set{a_n}_{n\in\NN}$ be Cauchy in $\RR$.
	Let $\varepsilon > 0$ be given.
	Since $\set{a_n}_{n\in\NN}$ is Cauchy there exists an $N_{\varepsilon/2}\in\NN$ such that $|a_n-a_m|<\varepsilon/2$ for all $n,m\geq N_{\varepsilon/2}$.
	Furthermore, $\set{a_n}_{n\in\NN}$ is bounded and thus by Bolzano-Weierstrass, there exists a $\set{a_{n_k}}_{k\in\NN}\subset\set{a_n}_{n\in\NN}$ such that $a_{n_k}$ converges to some $a\in\RR$.

	We claim that $a_n$ converges to $a$.
	Since $a_{n_k}$ converges to $a$, there exists a $K_{\varepsilon/2}\in\NN$ such that $|a_{n_k}-a|<\varepsilon/2$ for all $k\geq K_{\varepsilon/2}$.
	Take $N=\max\set{N_{\varepsilon/2},n_{K_{\varepsilon/2}}}$ and force $n,k$ large enough such that $n,n_k\geq N$.
	Consider $|a_n-a|$.
	\[
		|a_n-a|=|a_n-a_{n_k}+a_{n_k}-a|\leq |a_n-a_{n_k}|+|a_{n_k}-a|<\frac{\varepsilon}{2}+\frac{\varepsilon}{2}=\varepsilon
	\]
	Thus for any $\varepsilon >0$, there exists an $N\in\NN$ such that $|a_n-a|<\varepsilon$ for all $n\geq N$.
	Therefore $a_n$ converges to $a$ and $\RR$ is a complete metric space.
\end{proof}

\section{Bonus}

\begin{thm*}
	The set of infinite binary sequences is uncountable.
\end{thm*}

\begin{definition}[Countable Set]
	Let $X$ be a set.
	We say $X$ is \textit{countable} if and only if there exists a bijection between $X$ and a subset of $\NN$.
	If $X$ has a bijection with $\NN$ itself, then we say $X$ is \textit{countably infinite}.
\end{definition}

\begin{proof}
	Let $B$ represent the set of infinite binary sequences.
	Proceed via contradiction.
	Assume $B$ is countable.
	Then $B=\set{s_i}_{i\in\NN}$ where $s_i=(b_{i,1},b_{i,2},\ldots)$ with each $b_{i,j}\in\set{0,1}$.
	Define $\sim:\set{0,1}\rightarrow\set{0,1}$ by
	\[
		\sim x=
		\begin{cases}
			0 & x=1\\
			1 & x=0
		\end{cases}.
	\]
	We need to create a infinite binary sequence that is not equal to any $b_i$.
	Take $s=(\sim b_{1,1},\sim b_{2,2},\ldots)$.
	Obviously, $s\in B$.
	However, for each $i\in\NN$, $\sim b_{i,i}\neq b_{i,i}$ and thus $s\neq s_i$ for each $i\in\NN$.
	Therefore, $s\notin B$ which is a contradiction.
	Ergo, $B$ must be uncountable.
\end{proof}

\end{document}
