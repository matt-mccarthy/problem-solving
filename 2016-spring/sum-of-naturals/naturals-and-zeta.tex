\documentclass[]{simple}

\title{Sum of the Natural Numbers and the Riemann-Zeta Function}
\date{January, 2016}
\author{Matt McCarthy\\\href{mailto:matthew.mccarthy.12@cnu.edu}{matthew.mccarthy.12@cnu.edu}}

\addbibresource{naturals-and-zeta.bib}

\begin{document}

\maketitle

\section{Introduction}

We begin by introducing a rather interesting result, namely that the sum of the natural numbers is $-1/12$.
\begin{thm}\label{wrong}
	\[
		\sum_{n=1}^\infty n = -\frac{1}{12}
	\]
\end{thm}
We now provide the canonical proof as shown by \cite{wrong-pf}.
\begin{proof}[Proof (\autoref{wrong})]
	Define $s:=1+2+3+4+\ldots$, $s_1:=\sum_{n=0}^\infty(-1)^n$, and $s_2:=1-2+3-4+\ldots$.
	If we consider $s_1$, when we stop after an odd number of terms, the partial sum is $1$ but when we stop after an even number of terms, the partial sum is zero.
	Obviously, the series converges to the average of the two, thus $s_1=1/2$.
	Consider $2s_2$.
	\[
	\begin{array}{rrrrrrr}
		 &1 & -2 &+3 &-4&+\ldots\\
		+&  &1   &-2 &+3&-4&+\ldots\\\hline
		 &1 &-1  &+1 &-1&+\ldots
	\end{array}
	\]
	Since $2s_2=s_1=1/2$, $s_2=1/4$.
	Consider $s-s_2$.
	\[
		\begin{array}{rrrrrr}
			&1&+2&+3&+4&+\ldots\\
			-&1&-2&+3&-4&+\ldots\\\hline
			&0&+4&+0&+8&+\ldots
		\end{array}
	\]
	Moreover, we can factor out the four to get $s-s_2=4[1+2+3+\ldots]=4s$
	Solving for $s$ then yields $s=-1/12$ and completes the proof.
\end{proof}
\begin{analysis}
	The trick to this proof lies in exploiting the commutativity of addition.
	By doing so, we can manipulate the subtraction of two infinite series into a single alternating series.
	After that step, we then use the Taylor series we found in the lemma to complete the proof.
\end{analysis}

Now that we have ``shown'' that the natural numbers sum up to $-1/12$ the only things left to do are show that it is wrong and how this incorrect result came to be so prolific.

\section{Refutation}

We begin refuting this result by stating two of the limit laws and by providing the definition of an infinite series.
\begin{thm}[Limit Laws]
	Let $\set{x_n}_{n=1}^\infty$ and $\set{y_n}_{n=1}^\infty$ be real-valued sequences that converge to $x$ and $y$ respectively.
	Then all of the following hold.
	\begin{enumerate}
		\item The sequence $\set{x_n+y_n}_{n=1}^\infty$ converges to $x+y$.
		\item Let $k\in\RR$ then $\set{k\cdot x_n}_{n=1}^\infty$ converges to $k\cdot x$.
	\end{enumerate}
\end{thm}
These laws should seem familiar, considering that we used both in our ``proof'' of \autoref{wrong} however we may have invoked them without respecting their hypotheses.

\begin{definition}[Infinite Series]
	Let $\set{x_k}_{k=1}^\infty$ be a sequence of real numbers and let $x\in\RR$.
	An \textit{infinite series} is an expression of the form
	\[
		\sum_{k=1}^\infty x_k=x_1+x_2+\ldots.
	\]
	The corresponding \textit{sequence of partial sums} is defined by
	\[
		s_n:=\sum_{k=1}^n x_k.
	\]
	The series, $\sum_{k=1}^\infty x_k$ \textit{converges} to $x$ if and only if the sequence $\set{s_n}_{n=1}^\infty$ converges to $x$.
\end{definition}
This definition tells us that an infinite sum exists if and only if the its corresponding sequence of partial sums converges, which leads us to our biggest error: assuming the sum exists.

To determine whether or not the sum exists, we need to consider the sequence of partial sums,
\[
	s_n:=\sum_{k=1}^n k.
\]
If we inspect the sequence long enough, we see that the sum only increases and is what we call a \textit{monotone increasing sequence}.
\begin{definition}[Monotone Increasing Sequence]
	Let $\set{x_n}_{n=1}^\infty$ be a real-valued sequence.
	We say $\set{x_n}$ is \textit{monotone increasing} if and only if $x_n\leq x_{n+1}$ for each $n\in\NN$.
\end{definition}
If we consider the $(n+1)$ partial sum, we see
\[
	s_{n+1}=\sum_{k=1}^{n+1} k = \paren{\sum_{k=1}^{n} k} + (n+1) \geq \sum_{k=1}^{n} k=s_n.
\]
Thus, $\set{s_n}$ is a monotone increasing sequence.

Since $s_n$ is monotone increasing, we have a very nice theorem that turns our limit problem into a boundedness problem.

\begin{definition}[Upper Bound]
	Let $\set{x_n}_{n=1}^\infty$ be a real valued sequence and let $b\in\RR$.
	We say $b$ is an \textit{upper bound} of $\set{x_n}$ if and only if $x_n\leq b$ for each $n\in\NN$.
	If any such $b$ exists, we say $\set{x_n}$ is \textit{bounded above}.
\end{definition}

\begin{thm}[Monotone Convergence Theorem]
	Let $\set{x_n}_{n=1}^\infty$ be a real-valued monotone increasing sequence. Then $\set{x_n}$ converges if and only if $\set{x_n}$ is bounded above.
\end{thm}

With this theorem in our arsenal, we will now show that this infamous sum does not converge.

\begin{proposition}
	The sequence defined by
	\[
		s_n:=\sum_{k=1}^n k
	\]
	is unbounded.
\end{proposition}
\begin{proof}
Assume that $s_n$ is bounded above.
Ergo there exists a natural number $N > 0$ such that $s_n:=\sum_{k=1}^n k \leq N$ for every $n\in\NN$.
We know that
\[
	s_n:=\sum_{k=1}^n k = \frac{n(n+1)}{2}.
\]
Consider $s_{2N}$.
\[
	s_{2N} = \frac{2N(2N+1)}{2}=N(2N+1)=2N^2+N>N
\]
The previous statement contradicts our assumption and thus $s_n$ is an unbounded sequence.
\end{proof}
\begin{analysis}
	We perform a standard contradiction proof.
	The only tricks we use are the fact that
	\[
		1+2+\ldots+n=\frac{n(n+1)}{2}
	\]
	and a careful selection of $n$.
	Essentially, the proof boils down to picking a candidate for a bound and showing that we have a term in the sequence that is greater than our bound.
\end{analysis}

Since $s_n$ is unbounded, by the monotone convergence theorem, $s_n$ does not converge and neither does our sum.
Thus, we cannot invoke limit laws in order to compute $-4s$ and $s-4s$ in our proof of \autoref{wrong}.
In turn, this invalidates our derivation of $-3s$ which destroys the rest of the proof.

\section{The Riemann-Zeta Function}

To figure out why people believe this nonsense, we turn to the Riemann-Zeta function.
Historically, the Riemann-Zeta function arose as a way to extend the Euler-Zeta function, defined on real numbers greater than 1
\[
	\zeta(s)=\sum_{n=1}^\infty \frac{1}{n^s}
\]
to the rest of the complex plane \cite{edwards}.
When viewed in this light, one may think that
\[
	\zeta(-1)=\sum_{n=1}^\infty n.
\]
However, this is not the case since the Riemann-Zeta function is defined as follows.
\begin{definition}[Riemann-Zeta Function from \cite{edwards}]
	The \textit{Riemann-Zeta function} is the map $\zeta:\CC\rightarrow\CC$ defined by the contour integral
	\[
		\zeta(s):=\frac{\Pi(-s)}{2\pi i}\int_\gamma\frac{(-x)^s}{(e^x-1)x}dx
	\]
	where $\gamma$ is a curve that starts at $+\infty$, moves towards the origin along the positive real axis, circles the origin in a counterclockwise direction, and returns to $+\infty$ along the positive real axis.
	Furthermore, $\Pi(s)$ is defined as
	\[
		\Pi(s):=\int_0^\infty e^{-x}x^s dx.
	\]
\end{definition}

As we can see from the definition, this function does not resemble the geometric series in any way.
However, if $s$ is real and greater than $1$ (note: strictly greater than, not greater than or equal to) the zeta function can be written as
\begin{equation}\label{zeta-series}
	\zeta(s)=\sum_{n=1}^\infty\frac{1}{n^s}.
\end{equation}
Furthermore, since $-1$ is less than $1$,
\[
	\zeta(-1)\neq\sum_{n=1}^\infty n.
\]

Another interesting property of the Riemann-Zeta function pops up when we let $s$ be a negative integer.
\begin{proposition}[From \cite{edwards}]
	For any natural number $n$,
	\[
		\zeta(-n)=(-1)^n\frac{B_{n+1}}{n+1}
	\]
	where $B_{n+1}$ is the $n+1$th Bernoulli number.
\end{proposition}
By using this property we get
\begin{equation}\label{zeta-1}
	\zeta(-1)=(-1)\frac{B_2}{2}=(-1)\paren{\frac{1}{6}}\paren{\frac{1}{2}}=-\frac{1}{12}.
\end{equation}

Very likely, the incorrect conclusion that $1+2+3+\ldots=-1/12$ came out of some confusion arising from \autoref{zeta-series}, \autoref{zeta-1}, and the history of the zeta function.

\printbibliography

\end{document}
