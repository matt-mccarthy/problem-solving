\documentclass[notitlepage]{simple}

\title{Algebraic Properties of the Gaussian Integers}
\author{Matt McCarthy}
\date{April 2016}

\begin{document}

\maketitle

\begin{thm*}
	The Gaussian Integers, denoted $\ZZ(i)$, form a Euclidean domain.
\end{thm*}

\section{Background}

Before we can talk about Euclidean domains, we first need to introduce the definition of a ring.

\begin{definition}[Ring]
	Let $R$ be a nonempty set, and let $+:R^2\rightarrow R$ and $\cdot:R^2\rightarrow R$ be binary operations on $R$.
	Then we say $R$ is a \textit{ring} if all of the following hold.
	\begin{enumerate}
		\item The structure $(R,+)$ is an abelian group whose identity we denote as 0.
		\item For any $a,b,c\in R$, $a(bc)=(ab)c$ (Multiplicative Associativity).
		\item For any $a,b,c\in R$, $a(b+c)=ab+ac$ (Left Distributivity).
		\item For any $a,b,c\in R$, $(a+b)c=ac+bc$ (Right Distributivity).
	\end{enumerate}
	If one says $R$ is a ring, we imply that there exists some addition and some multiplication operators which we denote as $a+b$ and $ab$ respectively.
\end{definition}

\begin{definition}[Ring with Unity]
	Let $R$ be a ring.
	Then $R$ is a \textit{ring with unity} if there exists a $1\in R$ such that for any $a\in R$, $a\cdot 1 = 1\cdot a = a$.
	If such a $1$ exists, we call it the \textit{unity}.
\end{definition}

\begin{definition}[Commutative Ring]
	Let $R$ be a ring.
	Then we say $R$ is \textit{commutative} if for any $a,b\in R$, $ab=ba$.
\end{definition}

The integers, denoted $\ZZ$, are a commutative ring with unity because they satisfy all of the above properties under the usual addition and multiplication.
Another helpful definition is that of a subring.

\begin{definition}[Subring]
	Let $R$ be a ring and let $S$ be a nonempty subset of $R$.
	Then $S$ is a \textit{subring} of $R$ if $(S,+,\cdot)$ is also a ring.
\end{definition}

Furthermore, we have a test which makes it easier to show a subset is a subring.

\begin{proposition}[Subring Test]
	Let $R$ be a ring, and let $S\subseteq R$ be nonempty.
	Then $S$ is a subring of $R$ if and only if for any $a,b\in S$, $a-b$ and $ab$ are also in $S$.
\end{proposition}

Now we need a few more definitions and then we can proceed to proving the theorem.
First, we need to define what a zero divisor is.

\begin{definition}[Zero Divisor]
	Let $R$ be a ring and let $a\in R$ be nonzero.
	We say $a$ is a \textit{zero divisor} if there exists a nonzero $b\in R$ such that $ab = 0$.
\end{definition}

An example of a zero divisor is 2 in $\ZZ_6$, since $2\cdot 3\equiv 0\mod{6}$.
An important property of zero divisors is that they cannot be inverted.
Thus, if our ring has no zero divisors it is fairly nice; in fact it is nice enough that we name it.

\begin{definition}[Integral Domain]
	Let $R$ be a commutative ring with unity.
	Then we say $R$ is an \textit{integral domain} if $R$ has no zero-divisors.
\end{definition}

We call these structures integral domains, because they behave like the integers.
That is there is a unity, multiplication commutes, and we can multiply any nonzero elements together to get another nonzero element.

Next we will define one of the strongest structures in algebra, the field.

\begin{definition}[Field]
	Let $\FF$ be a commutative ring with unity.
	Then $\FF$ is a \textit{field}, if for each $a\in\FF\setminus\set{0}$, there exists a $a^{-1}\in\FF$ such that $aa^{-1}=1$.
\end{definition}

One field that we will use in our proof is the complex numbers, denoted $\CC$.
Lastly, we define Euclidean domains.

\begin{definition}[Euclidean Domain]
	Let $R$ be an integral domain.
	Then we say $R$ is a \textit{Euclidean domain} if there exists a function $d:R\rightarrow(\ZZ^+\cup\set{0})$ such that
	\begin{enumerate}
		\item for any $x,y\in R\setminus\set{0}$, $d(xy)\geq d(x)$,
		\item and there exist $q,r\in R$ where $x=yq+r$ with $r=0$ or $d(r) < d(y)$.
	\end{enumerate}
	Any such $d$ is called a \textit{measure}.
\end{definition}
Essentially, Euclidean domains are rings where the division algorithm works.

\section{Solution}

To start, we define $\ZZ(i)$, the Gaussian Integers.

\begin{definition}[Gaussian Integers]
	The \textit{Gaussian Integers} are
	\[
		\ZZ(i)=\set{a+bi|a,b\in\ZZ}.
	\]
\end{definition}

We first need to show that $\ZZ(i)$ is an integral domain.

\begin{lemma}
	$\ZZ(i)$ is an integral domain under standard complex addition and multiplication.
\end{lemma}
\begin{proof}
	We know that $\CC$ is a field, therefore it is a commutative ring with identity and no zero divisors.
	Thus, it suffices to show that $\ZZ(i)$ is a subring of $\CC$ that contains 1.
	Since $1,0\in\ZZ$, $1+0i=1\in\ZZ(i)$.
	Thus $\ZZ(i)$ is nonempty since it contains the unity.
	We now need to show that for any $z=a+bi,w=c+di\in\ZZ(i)$, $z-w,zw\in\ZZ(i)$.
	We know that $z-w=(a-c)+(b-d)i$ and $zw=(ac-bd)+(ad+bc)i$.
	Since $\ZZ$ is a ring, $a-c,b-d,ac-bd$, and $ad+bc$ are in $\ZZ$ by closure.
	Therefore, $z-w,zw\in\ZZ(i)$.
	Thus $\ZZ(i)$ is a commutative ring with unity that has no zero divisors.
	Hence, $\ZZ(i)$ is an integral domain.
\end{proof}

Since $\ZZ(i)$ is an integral domain, we can embed it in what we call the \textit{field of fractions}, otherwise known as $\QQ(i)$.
We will assume that $\QQ(i)=\set{a+bi|a,b\in\QQ}$, which is true but requires a significant amount of background to show.
The proof of the following theorem hinges upon the previous assumption.

\begin{thm}
	$\ZZ(i)$ is a Euclidean domain.
\end{thm}
\begin{proof}
	In order to show that an integral domain is a Euclidean domain, we need to propose a measure.
	We claim that $d:\ZZ(i)\rightarrow\ZZ^{+}\cup\set{0}$ given by $d(z)=|z|^2$ is such a measure.

	To start we need to show that for any $z,w\in\ZZ(i)\setminus\set{0}$, $d(zw)\geq d(z)$.
	We know that $d(zw)=|zw|^2$.
	However, from Euler's formula, we know that $|zw|=|z||w|$.
	Therefore, $d(zw)=|z|^2|w|^2$.
	Furthermore, by Euler's formula, the only element with modulus less than 1 in $\ZZ(i)$ is 0.
	Therefore, $d(zw)\geq |z|^2=d(z)$.

	Next, we need to find $q,r\in\ZZ(i)$ such that $z=wq+r$ where $r=0$ or $d(r) < d(w)$.
	To do so, we embed $\ZZ(i)$ is $\QQ(i)$ and consider $z/w$.
	We know $z/w=\alpha+\beta i$ with $\alpha,\beta\in\QQ$.
	Let $\alpha',\beta'$ be the nearest integers to $\alpha$ and $\beta$ respectively.
	Then $|\alpha-\alpha'|\leq 1/2$ and $|\beta-\beta'|\leq 1/2$.
	Furthermore,
	\[
		\frac{z}{w}=\alpha-\alpha'+\alpha' + (\beta-\beta'+\beta')i=(\alpha'+\beta' i)+((\alpha-\alpha')+(\beta-\beta')i).
	\]
	Solving for $z$ yields,
	\[
		z=(\alpha'+\beta' i)w+((\alpha-\alpha')+(\beta-\beta')i)w.
	\]
	Since $\alpha',\beta'\in\ZZ$, we know that $\alpha'+\beta' i\in\ZZ(i)$.
	Thus,
	\[
		((\alpha-\alpha')+(\beta-\beta')i)w=z-(\alpha'+\beta' i)w\in\ZZ(i)
	\]
	by closure.
	Take $q=\alpha'+\beta' i$ and $r=((\alpha-\alpha')+(\beta-\beta')i)w$.
	If $r=0$, we are done, otherwise consider $d(r)$.
	\[
		d(r)=|(\alpha-\alpha')+(\beta-\beta')i|^2d(w)=(|\alpha-\alpha'|^2+|\beta-\beta'|^2)d(w)
	\]
	However, we know that $|\alpha-\alpha'|\leq 1/2$ and $|\beta-\beta'|\leq 1/2$.
	Therefore,
	\[
		d(r)\leq \paren{\frac{1}{4}+\frac{1}{4}}d(w)=\frac{1}{2}d(w) < d(w)
	\]
	and $d$ is a measure on $\ZZ(i)$.
	Thus $\ZZ(i)$ is Euclidean.
\end{proof}

\end{document}
