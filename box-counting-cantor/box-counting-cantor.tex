\documentclass[notitlepage]{simple}

\author{Matt McCarthy}
\title{Box Counting Dimension of the Middle-$\lambda$ Cantor Set}
\date{April 2016}

\def\card{\operatorname{card}}
\def\dim{\operatorname{dim}}
\def\diam{\operatorname{diam}}

\begin{document}
	\maketitle

	\begin{thm*}
		Let $C$ denote the middle-$\lambda$ Cantor set with $0 < \lambda < 1$.
		Then the box counting dimension of $C$ is
		\[
			\dim_B C = \frac{\ln 2}{\ln\paren{\frac{2}{1-\lambda}}}.
		\]
	\end{thm*}

	\section{Background}

	\subsection{Fractal Analysis}

	We begin by defining a $\delta$-cover of a set.
	\begin{definition}
		Let $X\subset\RR^n$ and $\delta>0$.
		Then a \textit{$\delta$-cover} of $X$ is a set $D=\set{D_i}_{i=1}^{n_D}$ such that $X\subseteq \cup_{D_i\in D} D_i$ and $\diam D_i \leq \delta$ for all $D_i\in D$.
		We denote the collection of all such covers as $\mathcal{D}_\delta(X)$.
	\end{definition}

	Now that we have $\delta$-covers, we can define the \textit{box-counting dimension} of a set.

	\begin{definition}
		Let $X$ be a subset of $\RR^n$.
		Then the \textit{box-counting dimension} of $X$, denoted $\dim_B(X)$, is defined as
		\[
			\lim\limits_{\delta\rightarrow0} \frac{\ln N_\delta(X)}{-\ln\delta}
		\]
		where
		\[
			N_\delta(X) = \min\limits_{D\in\mathcal{D}_\delta(X)} \card{D}.
		\]
	\end{definition}
	Note that the box-counting dimension of a set does not necessarily exist, however when it does exist it is usually easier to find.

	\begin{thm}[Squeeze Theorem]
		Let $(a_n),(b_n),(c_n)$ be real valued sequences such that $\lim a_n = \lim c_n = x$ and $a_n\leq b_n\leq c_n$ for all $n$ greater than some $n\in\NN$.
		Then $b_n$ converges to $x$.
	\end{thm}

	\subsection{The Cantor Set}

	We will now talk about the middle third Cantor set.
	Begin by defining $C_0=[0,1]$.
	We now remove the middle third from $C_0$ yielding, $C_1=[0,1/3]\cup[2/3,1]$.
	We then remove the middle third from \textit{each} subinterval of $C_1$, yielding $C_2 = [0,1/9]\cup[2/9,3/9]\cup [6/9,7/9]\cup[8/9,1]$.
	We iterate this process by removing the middle-third from each subinterval of $C_n$ and labeling the remaining set $C_{n+1}$.
	Finally, we define the Cantor set as $C=\cap_{n\in\NN} C_n$.
	We can similarly construct the middle-$\lambda$ Cantor set by performing the same construction but removing $\lambda$ instead of one third in each step.
	Furthermore, $C$ is an uncountable set that has no length left to it.

	\section{Solution}

	\begin{thm}
		Let $C$ denote the middle-$\lambda$ Cantor set with $0 < \lambda < 1$.
		Then the box counting dimension of $C$ is
		\[
			\dim_B C = \frac{\ln 2}{\ln\paren{\frac{2}{1-\lambda}}}.
		\]
	\end{thm}
	\begin{proof}
		Consider removing an interval of length $a\lambda$ from the middle of the interval $[0,a]$ for some $a>0$.
		Since we have removed $a\lambda$ from the interval, the length of this new set is exactly $a-a\lambda=a(1-\lambda)$.
		Moreover, since we removed it from the middle of the interval, this length is equally distributed among the two resultant subintervals.
		Thus the subintervals must have length $a((1-\lambda)/2)$.
		Furthermore, the leftmost interval must be $[0,a((1-\lambda)/2)]$.

		We now need to find the length of any subinterval of $C_n$.
		Since $C_0=[0,1]$, $l_0=1$.
		This tells us that $l_1=(1-\lambda)/2$ and that the leftmost interval in $C_1$ is $[0,(1-\lambda)/2]$.
		We know from the above logic that $l_{n+1}=l_n(1-\lambda)/2$ and its leftmost interval will be $[0,l_n(1-\lambda)/2]$.
		Solving this recursion yields that
		\[
			l_n = \paren{\frac{1-\lambda}{2}}^n
		\]
		and the leftmost interval of $C_n$ is $[0,((1-\lambda)/2)^n]$.
		Thus if we were to cover $C_n$ we would need $2^n$ sets of diameter $((1-\lambda)/2)^n$.

		Suppose $\delta >0$.
		Thus, we can find an $n$ such that $l_{n+1}\leq \delta < l_n$.
		Since $\delta < l_n$, we need no fewer than $2^n$ sets of diameter $\delta$ to cover $C_n\supset C$.
		Since $\delta \geq l_{n+1}$, we need no more than $2^{n+1}$ sets of diameter $\delta$ to cover $C_{n+1}\supset C$.
		Thus, we can glean the following inequality.
		\[
			2^{n} \leq N_\delta(C) \leq 2^{n+1}.
		\]
		Since $\ln$ is a monotone increasing function, we get
		\begin{equation}\label{ln-num}
			n\ln 2 \leq \ln N_\delta(C) \leq (n+1)\ln 2.
		\end{equation}
		Furthermore, since $l_{n+1}\leq \delta < l_n$,
		\begin{equation}\label{ln-den}
			\frac{1}{-\ln l_{n+1}} \leq \frac{1}{-\ln\delta} \leq \frac{1}{-\ln l_n}.
		\end{equation}

		Take \autoref{ln-num} and multiply it by $1/(-\ln\delta)$.
		\[
			\frac{n\ln 2}{-\ln\delta} \leq \frac{\ln N_\delta(C)}{-\ln\delta} \leq \frac{(n+1)\ln 2}{-\ln\delta}
		\]
		By \autoref{ln-den}, we have
		\[
			\frac{n\ln 2}{-\ln l_{n+1}} \leq \frac{\ln N_\delta(C)}{-\ln\delta} \leq \frac{(n+1)\ln 2}{-\ln l_n}.
		\]
		Consider $\ln l_n$.
		\[
			\ln l_n = \ln\paren{\paren{\frac{1-\lambda}{2}}^n} = -n\ln\paren{\frac{2}{1-\lambda}}
		\]
		Thus, our inequality becomes,
		\[
			\frac{n\ln 2}{(n+1)\ln\paren{\frac{2}{1-\lambda}}} \leq \frac{\ln N_\delta(C)}{-\ln\delta} \leq \frac{(n+1)\ln 2}{n\ln\paren{\frac{2}{1-\lambda}}}.
		\]
		If we want to find $\lim\limits_{\delta\rightarrow 0} \ln N_\delta(C)/(-\ln\delta)$, we need to find the limits of the far left and far right as $n\rightarrow\infty$.
		On the lefthand side we get
		\[
			\frac{n\ln 2}{(n+1)\ln\paren{\frac{2}{1-\lambda}}} = \frac{n\ln 2}{n\ln\paren{\frac{2}{1-\lambda}}+\ln\paren{\frac{2}{1-\lambda}}}\rightarrow\frac{\ln 2}{\ln\paren{\frac{2}{1-\lambda}}}
		\]
		by l'H\^opital's rule.
		On the righthand side we get
		\[
			\frac{(n+1)\ln 2}{n\ln\paren{\frac{2}{1-\lambda}}} = \frac{n\ln 2+\ln 2}{n\ln\paren{\frac{2}{1-\lambda}}}\rightarrow\frac{\ln 2}{\ln\paren{\frac{2}{1-\lambda}}}
		\]
		again by l'H\^opital's rule.
		Thus by squeeze theorem, $\dim_B (C)=\frac{\ln 2}{\ln\paren{\frac{2}{1-\lambda}}}$.
	\end{proof}
\end{document}
