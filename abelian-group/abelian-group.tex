\documentclass[notitlepage]{simple}

\author{Matt McCarthy}
\date{February 2016}
\title{Test for Abelian Groups}

\begin{document}
	\maketitle
	\begin{definition}
		Let $S$ be a set.
		Then a \textit{binary operation on $S$} is a map $*:S\times S\rightarrow S$.
	\end{definition}
	\begin{definition}
		Let $G$ be a set and $*$ be a binary operation on $G$.
		Then $(G,*)$ is a \textit{group} if and only if all of the following hold.
		\begin{enumerate}
			\item For each $a,b,c\in G$, $(a*b)*c=a*(b*c)$.
			\item There exists a $1\in G$ such that for every $a\in G$ $1* a=a* 1=a$.
			\item For each $a\in G$ there exists a $a^{-1}\in G$ such that $a*a^{-1}=a^{-1}*a=1$.
		\end{enumerate}
	\end{definition}
	\begin{definition}
		Let $(G,\cdot)$ be a group.
		Then we say $G$ is \textit{abelian} if and only if for each $a,b\in G$ $ab=ba$.
	\end{definition}
	\begin{thm}
		Suppose $(G,\cdot)$ is a group such that there exists an $n\in\NN$ where $(ab)^n=a^nb^n$, $(ab)^{n+1}=a^{n+1}b^{n+1}$, and $(ab)^{n+2}=a^{n+2}b^{n+2}$ for each $a,b\in G$.
		Then $G$ is abelian.
	\end{thm}
	\begin{proof}
		Consider $a^{n+1}b^{n+1}$.
		\[
			a^{n+1}b^{n+1} = (ab)^{n+1}=(ab)^n(ab)=a^nb^nab
		\]
		If we multiply on the right by $a^{-n}$ and the left by $b^{-1}$, we get
		\[
			ab^n=b^na.
		\]
		Consider $a^{n+2}b^{n+2}$.
		\[
			a^{n+2}b^{n+2} = (ab)^{n+1}=(ab)^{n+1}(ab)=a^{n+1}b^{n+1}ab
		\]
		Again, multiplying on the right $a^{-(n+1)}$ and on the right by $b^{-1}$ yields,
		\[
			ab^{n+1}=b^{n+1}a.
		\]
		We now do some manipulation.
		\begin{align*}
			ab^{n+1}&=b^{n+1}a\\
			&=b(b^na)\\
			&=b(ab^n)
		\end{align*}
		Multiplying on the right by $b^{-n}$ yields,
		\[
			ab=ba
		\]
		and thus $G$ is abelian.
	\end{proof}
\end{document}
