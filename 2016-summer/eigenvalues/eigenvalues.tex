\documentclass[notitlepage]{problem-solving}

\author{Matt McCarthy}
\date{June 2016}
\title{Eigenvalues and Eigenvectors}

\begin{document}

\maketitle

\begin{problem*}
	Let $\tau:\FF_5^3\rightarrow\FF_5^3$ be the endomorphism defined by
	\[
		\tau =
		\begin{pmatrix}
			1 & 0 & 2\\
			2 & 0 & 1\\
			1 & 2 & 1
		\end{pmatrix}.
	\]
	Find all eigenvalues and eigenvectors of $\tau$.

	(Note: $\FF_5$ denotes $\ZZ/5\ZZ$, the integers modulo 5.)
\end{problem*}

\section{Background}

Lets begin with the definition of eigenvalues and eigenvectors.
\begin{definition}
	Let $V$ be a vector space over a field $\FF$ and let $\tau$ be an endomorphism on $V$.
	A scalar $\lambda\in\FF$ is an \textit{eigenvalue} of $\tau$ if there exists a nonzero vector $v$ such that
	\[
		\tau v =\lambda v.
	\]
	If such a $v$ exists, it is called an \textit{eigenvector} of $\tau$ associated with $\lambda$.
\end{definition}

Furthermore, in finite-dimensional vector spaces, each endomorphism has a minimal polynomial.
\begin{definition}
	Let $V$ be a finite-dimensional vector space over a field $\FF$ and let $\tau$ be an endomorphism on $V$.
	Then the \textit{minimal polynomial} of $\tau$ is the generator of the ideal
	\[
		I_\tau = \set{p\in\FF[x] | p(\tau) = 0}.
	\]
\end{definition}

Moreover, the eigenvalues of an endomorphism are the zeros of this minimal polynomial.
\begin{thm}
	Let $\tau$ be an endomorphism with minimal polynomial $p$.
	Then the set of zeros of $p$ and the set of eigenvalues of $\tau$ are equal.
\end{thm}

\section{Solution}

\begin{problem*}
	Let $\tau:\FF_5^3\rightarrow\FF_5^3$ be the endomorphism defined by
	\[
		\tau =
		\begin{pmatrix}
			1 & 0 & 2\\
			2 & 0 & 1\\
			1 & 2 & 1
		\end{pmatrix}.
	\]
	Find all eigenvalues and eigenvectors of $\tau$.

	(Note: $\FF_5$ denotes $\ZZ/5\ZZ$, the integers modulo 5.)
\end{problem*}

We begin by finding the minimal polynomial of $\tau$.
To do so, we apply $\tau$ multiple times to the vector $e_1 = (1,0,0)$.
\[
	\tau^0 e_1 =
	\begin{pmatrix}
		1\\
		0\\
		0
	\end{pmatrix}
	\hspace{1cm}
	\tau^1 e_1 =
	\begin{pmatrix}
		1 & 0 & 2\\
		2 & 0 & 1\\
		1 & 2 & 1
	\end{pmatrix}
	\begin{pmatrix}
		1\\
		0\\
		0
	\end{pmatrix}
	=
	\begin{pmatrix}
		1\\
		2\\
		1
	\end{pmatrix}
	\hspace{1cm}
	\tau^2 e_1 =
	\begin{pmatrix}
		1 & 0 & 2\\
		2 & 0 & 1\\
		1 & 2 & 1
	\end{pmatrix}
	\begin{pmatrix}
		1\\
		2\\
		1
	\end{pmatrix}
	=
	\begin{pmatrix}
		3\\
		3\\
		1
	\end{pmatrix}
\]
We now need to check that $B=\set{e_1, \tau e_1, \tau^2 e_1}$ is linearly independent.
Therefore, we must solve the equation
\[
	a(1,0,0)+b(1,2,1)+c(3,3,1)=0.
\]
Indeed, this equation implies that $a=b=c=0$.
Ergo $B$ is linearly independent.
Next, we compute one more power of $\tau$.
\[
\tau^3 e_1 =
\begin{pmatrix}
	1 & 0 & 2\\
	2 & 0 & 1\\
	1 & 2 & 1
\end{pmatrix}
\begin{pmatrix}
	3\\
	3\\
	1
\end{pmatrix}
=
\begin{pmatrix}
	0\\
	2\\
	0
\end{pmatrix}
\]
Consider the equation
\[
	a(1,0,0)+b(1,2,1)+c(3,3,1)=(0,2,0).
\]
This implies that $a=1$, $b=3$, and $c=2$.
Thus,
\[
	\tau^3e_1=e_1 + 3\tau e_1 + 2 \tau^2 e_1
\]
which yields a minimal polynomial of
\[
	\mu_{\tau, e_1} = X^3 +3X^2 + 2X +4.
\]
Recall that the set of eigenvalues of a linear transform is the set of zeros of its minimal polynomial.
We notice that $\mu_{\tau,e_1}(1)=0$, thus $X-1|\mu_{\tau,e_1}$.
Therefore,
\[
	\mu_{\tau,e_1} = (X-1)(X^2+4X+1).
\]
When we apply the quadratic formula, we get that
\[
	X= 3 + 3\sqrt{2}, 3+2\sqrt{2}
\]
are also zeros of $\mu_{\tau,e_1}$.
Therefore the eigenvalues of $\mu_{\tau,e_1}$ are $1, 3+3\sqrt{2}, 3+2\sqrt{2}$.
To find the eigenvectors that correspond to 1, we solve the following equation.
\[
\begin{pmatrix}
	1 & 0 & 2\\
	2 & 0 & 1\\
	1 & 2 & 1
\end{pmatrix}
\begin{pmatrix}
	a\\
	b\\
	c
\end{pmatrix}
=
1\cdot
\begin{pmatrix}
	a\\
	b\\
	c
\end{pmatrix}
\]
This corresponds to the following system of equations.
\begin{align*}
	a + 0b + 2c &= a\\
	2a + 0b + c &= b\\
	a + 2b + c &= c
\end{align*}
This system implies that $c = 0$ and $b = 2a$, thus any vector in $\FF_5^3$ of the form $(a,2a,0)$ is an eigenvector of 1.

Even though $\sqrt{2}\notin\FF_5$, we can embed $\FF_5$ inside $\FF_5[\sqrt{2}]$.
By performing a similar process we get that any vector of the form $(a,3\sqrt{2} a, (1+4\sqrt{2})a)$ is an eigenvector of $3+3\sqrt{2}$ and $(a,2\sqrt{2} a, (1+\sqrt{2})a)$ is an eigenvector of $3+2\sqrt{2}$ where $a\in\FF_{5}[\sqrt{2}]$.

\end{document}
