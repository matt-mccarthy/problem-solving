\documentclass[notitlepage]{problem-solving}

\author{Matt McCarthy}
\title{Fibonacci Generating Function}
\date{June 2016}

\addbibresource{fibonacci.bib}
\nocite{*}

\begin{document}

\maketitle

\begin{problem*}
	Let $(F_n)_{n\in\NN}$ denote the sequence of Fibonacci numbers.
	Find the closed form of the generating function for $F_n$ and use it to create a closed form for $F_n$.
\end{problem*}

\section{Background}

Lets begin with the definition of the Fibonacci numbers.
\begin{definition}
	The Fibonacci numbers $(F_n)_{n\in\NN}$ are defined by $F_1=F_2=1$, and $F_{n+2}=F_{n+1}+F_n$.
\end{definition}

From the Fibonacci numbers we get the golden ratio, $\varphi$, and its conjugate, $\overline{\varphi}$.
\[
	\varphi = \frac{1+\sqrt{5}}{2} \hspace{1cm} \overline{\varphi} = \frac{1-\sqrt{5}}{2}
\]

While the Fibonacci recurrence is simple, its difficult to tell what the $n$th term is without computing all of the terms before it.
Since the Fibonacci's are defined recursively, we can use \textit{generating functions} to extract a closed form for $F_n$ relatively easily.

\begin{definition}
	Let $(a_n)_{n\in\NN}$ be a $\RR$-valued sequence.
	Then the generating function for $a_n$ is the power series
	\[
		G(x) = \sum_{n\in\NN} a_nx^n.
	\]
\end{definition}

After we create the generating function, we generally want to find its closed form (that is, a representation without using limits).
Then, we use that new representation to find a closed form for the terms of the sequence.

\section{Solution}

\begin{problem*}
	Let $(F_n)_{n\in\NN}$ denote the sequence of Fibonacci numbers.
	Find the closed form of the generating function for $F_n$ and use it to create a closed form for $F_n$.
\end{problem*}

We begin by setting up a generating function, $F(x)$, for $F_n$.
\[
	F(x) = \sum_{n=1}^\infty F_n x^n.
\]
The next step is to invoke the recurrence, however $F_1$ and $F_2$ need to be handled separately since they are not defined recursively.
Thus,
\[
	F(x) = F_1 x + F_2 x^2 + \sum_{n=3}^\infty F_n x^n.
\]
Now we invoke the definition of the Fibonacci numbers yielding,
\[
	F(x) = x + x^2 + \sum_{n=3}^\infty (F_{n-1} + F_{n-2}) x^n.
\]
By rearranging we get,
\[
	F(x) = x + x^2 +\paren{\sum_{n=3}^\infty F_{n-1} x^n} + \paren{\sum_{n=3}^\infty F_{n-2} x^n}.
\]
We then factor out as many powers of $x$ from the sums and reindex to get
\[
	F(x) = x + x^2 +x\paren{F_1x-F_1x+\sum_{n=2}^\infty F_{n} x^n} + x^2\paren{\sum_{n=1}^\infty F_{n} x^n}.
\]
A little algebraic manipulation yields,
\[
	F(x) = x+x^2 -x^2 +x\paren{\sum_{n=1}^\infty F_n x^n} + x^2 \paren{\sum_{n=1}^\infty F_n x^n}.
\]
Note, that we see the original definition of $F(x)$ in this new form, therefore
\[
	F(x) = x + xF(x) + x^2 F(x).
\]
We now solve for $F(x)$, yielding a closed form of
\[
	F(x) = \frac{x}{1-x-x^2}.
\]

The next step is to find the closed form of $F_n$.
We know that $\varphi$ and $\overline{\varphi}$ are roots of  the polynomial in the denominator, therefore
\[
	F(x)=\frac{x}{(1-\varphi x)(1-\overline{\varphi}x)}.
\]
To do so, we perform partial fraction decomposition of $F(x)$ yielding,
\[
	F(x) = \frac{1}{\sqrt{5}} \paren{\frac{1}{1-\varphi x}-\frac{1}{1-\overline{\varphi}x}}.
\]
Using the formula for the power series for $(1-x)^{-1}$ yields,
\[
	F(x) = \sum_{k=1}^\infty \frac{\varphi^n-\overline{\varphi}^n}{\sqrt{5}} x^n = \sum_{k=1}^\infty F_n x^n.
\]
By uniqueness of power series, we deduce that
\[
	F_n = \frac{\varphi^n-\overline{\varphi}^n}{\sqrt{5}}.
\]

\printbibliography

\end{document}
