\documentclass[notitlepage]{problem-solving}

\def\ord{\operatorname{ord}}

\title{Sums Convergent under the $p$-adic Norm}
\date{June 2016}
\author{Matt McCarthy}

\addbibresource{p-adic-sum.bib}
\nocite{*}

\begin{document}

\maketitle

\begin{problem*}
	Show that
	\[
		\sum_{n=0}^\infty 2^n = -1
	\]
	under the 2-adic metric.
\end{problem*}

\section{Background}

In order to talk about limits, we first need to understand the concept of a metric.
\begin{definition}[Metric Space]
	Let $X$ be a non-empty set and let $d:X\times X\rightarrow \RR$ be a function.
	Then $d$ is a \textit{metric} on $X$ if all of the following hold.
	\begin{enumerate}
		\item For all $x,y\in X$, $d(x,y)\geq 0$.
		\item For all $x,y\in X$, $d(x,y) = 0$ iff $x=y$.
		\item For all $x,y\in X$, $d(x,y)=d(y,x)$.
		\item For all $x,y,z\in X$, $d(x,z)\leq d(x,y)+d(y,z)$ (triangle inequality).
	\end{enumerate}
	If $d$ is a metric on $X$, then we say $(X,d)$ forms a \textit{metric space}.
\end{definition}
After defining the metric space, we can consider whether or not a sequence in that space converges.
\begin{definition}[Convergent Sequence]
	Let $(X,d)$ be a metric space and let $(a_n)_{n\in\NN}\subseteq X$ and let $a\in X$.
	We say $a_n$ is a \textit{converges} to $a$ if for all $\varepsilon>0$, there exists an $N\in\NN$ such that $d(a_n,a)<\varepsilon$ for all $n\geq N$.
	If such an $a$ exists, we say $a_n$ is \textit{convergent in $X$}.
\end{definition}
Furthermore, if a sequence is convergent in a metric space, it is also Cauchy in that space.
\begin{definition}[Cauchy Sequence]
	Let $(X,d)$ be a metric space and let $(a_n)_{n\in\NN}\subseteq X$.
	We say $a_n$ is a \textit{Cauchy sequence} if for all $\varepsilon>0$, there exists an $N\in\NN$ such that $d(a_n,a_m)<\varepsilon$ for all $n,m\geq N$.
\end{definition}
While convergent implies Cauchy, the other way does not always hold.
For example in $\QQ$ under the Euclidean metric ($d(x,y)=|x-y|$), the sequence defined by $(1+1/n)^n$ is Cauchy but not convergent.
However, if we move to $\RR$, $(1+1/n)^n$ converges to $e$.
Cauchy sequences are sequences that \textit{should} be convergent in our space.
If they are not, then we need to move to what is called the completion of the metric space.
\begin{definition}[Complete Metric Space]
	Let $(X,d)$ be a metric space.
	We say $X$ is \textit{complete} if all Cauchy sequences in $X$ converge in $X$.
\end{definition}
\begin{thm}
	Let $(X,d)$ be a metric space.
	Then $X$ has a unique completion, $C(X,d)$, up to isometry.
\end{thm}

Now that we have enough background in analysis, lets talk about the $p$-adic numbers.
Before we can define the $p$-adic's, we need to introduce the $p$-adic ordinal and $p$-adic absolute value first.
\begin{definition}[$p$-adic Ordinal]
	Let $p$ be a prime and let $a\in\ZZ$ be nonzero.
	Then the \textit{$p$-adic ordinal} of $a$, denoted $\ord_p a$, is defined as
	\[
		\ord_p a = \max\set{n\text{ s.t. } p^n | a}.
	\]
	Furthermore, for any nonzero $x=b/c\in\QQ$,
	\[
		\ord_p x = \ord_p a - \ord_p b.
	\]
\end{definition}
Using the definition of $p$-adic ordinal, we now provide the definition of the $p$-adic absolute value.
\begin{definition}[$p$-adic absolute value]
	Let $p$ be a prime and $x\in\QQ$.
	Then the \textit{$p$-adic norm of $x$}, denoted $|x|_p$, is defined as
	\[
		|x|_p =
		\begin{cases}
			p^{-\ord_p x} & x\neq 0\\
			0 & x=0.
		\end{cases}
	\]
\end{definition}
While an absolute value is not a metric in and of itself, it generates a metric.
Just like the regular absolute value generates the metric $d(x,y)=|x-y|$, the $p$-adic absolute value uses $|x-y|_p$ as a metric.
With this, we can define the $p$-adic numbers.
\begin{definition}[$p$-adic Numbers]
	The set of \textit{$p$-adic numbers} is the completion of the metric space $(\QQ,|\cdot|_p)$.
\end{definition}
Additionally, $\QQ_p$ satisfies a stronger version of the triangle inequality, that is the distance between any two elements is no more than the maximum of the distances to a third element.
\begin{thm}[Strong Triangle Inequality]
	For all $x,y\in\QQ_p$, $|x+y|_p\leq\max\set{|x|_p,|y|_p}$.
\end{thm}

\section{Solution}

\begin{problem*}
	Show that
	\[
		\sum_{n=0}^\infty 2^n = -1
	\]
	under the 2-adic norm.
\end{problem*}

To start our solution, we prove that if a sequence in $\QQ_p$ converges to 0 under the $p$-adic metric, the infinite sum of all of its terms is convergent in $\QQ_p$.

\begin{thm}
	Let $(a_n)_{n\in\NN}\subseteq\QQ_p$ such that $a_n\rightarrow 0$.
	Then $\sum_{n=0}^\infty a_n$ converges in $\QQ_p$.
\end{thm}
\begin{proof}
	Let $\varepsilon > 0$ be given.
	Since $a_n\rightarrow 0$, there exists an $N\in\NN$ such that $|a_n|<\varepsilon$.
	Force $n\geq m\geq N$.
	Consider $|\sum_{k=0}^n a_k - \sum_{k=0}^m|_p$.
	\[
		\abs{\sum_{k=0}^n a_k - \sum_{k=0}^m}_p a_k = \abs{\sum_{k=m+1}^n a_k}_p
	\]
	Thus, by the strong triangle inequality,
	\[
		\abs{\sum_{k=0}^n a_k - \sum_{k=0}^m a_k}_p \leq \max\limits_{m+1\leq k\leq n}\set{|a_k|_p}.
	\]
	However, we know that for each $k\geq N$, $|a_k|_p<\varepsilon$.
	Therefore,
	\[
		\abs{\sum_{k=0}^n a_k - \sum_{k=0}^m a_k}_p < \varepsilon
	\]
	and $\paren{\sum_{k=0}^n a_k}_{n\in\NN}$ is Cauchy in $\QQ_p$.
	Since $\QQ_p$ is defined as a complete metric space, $\paren{\sum_{k=0}^n a_k}_{n\in\NN}$ converges in $\QQ_p$.
\end{proof}

Now, in order to show that $\sum_{n=0}^\infty 2^n$ converges with respect to the $p$-adic metric, we show that $2^n\rightarrow 0$ with respect to the $p$-adic metric.

\begin{lemma}
	Under the 2-adic metric, $2^n\rightarrow 0$.
\end{lemma}
\begin{proof}
	Let $\varepsilon > 0$ be given.
	Without loss of generality, assume $\varepsilon < 1$.
	Thus, $\lg\varepsilon < 0$.
	Take $N > -\lg\varepsilon$ and let $n\geq N$.
	Then
	\[
		|2^n|_2 = 2^{-n} \leq 2^{-N} < 2^{\lg\varepsilon} = \varepsilon.
	\]
	Therefore, $2^n\rightarrow 0$ under $|\cdot|_2$.
\end{proof}

From here, we do some algebraic manipulation to find the limit point.

\begin{proposition}
	In $\QQ_2$,
	\[
		\sum_{n=0}^\infty 2^n = -1.
	\]
\end{proposition}
\begin{proof}
	Since $2^n\rightarrow 0$, $\sum_{n=0}^\infty 2^n$ converges to some $S\in\QQ_p$.
	Consider $S$.
	\[
		S= \sum_{n=0}^\infty 2^n = 1 + \sum_{n=1}^\infty 2^n = 1+ 2\sum_{n=1}^\infty 2^{n-1} = 1+ 2 \sum_{n=0} 2^n = 1+ 2S
	\]
	Solving for $S$ yields,
	\[
		S = -1.
	\]
\end{proof}

\printbibliography

\end{document}
