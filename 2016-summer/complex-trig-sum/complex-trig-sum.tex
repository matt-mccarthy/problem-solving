\documentclass[notitlepage]{problem-solving}

\author{Matt McCarthy}
\title{Trigonometric Sum and Difference Formulas in $\CC$}
\date{June 2016}

\addbibresource{complex-trig-sum.bib}
\nocite{*}

\begin{document}

\maketitle

\begin{problem*}
	Show that for any $z,w\in\CC$,
	\[
		\sin(z+w)=\cos z\sin w+\sin z\cos w \text{ and } \cos(z+w)=\cos z\cos w - \sin z\sin w.
	\]
\end{problem*}

\section{Background}

In the world of $\CC$, differentiability is a stronger condition than in $\RR$.
To be precise, once a function is differentiable in $\CC$, it is infinitely differentiable.
This happens because differentiability has more stringent requirements in $\CC$.
In order for a $\CC$-valued function to be differentiable on a domain, it must be \textit{analytic} on that domain.
\begin{definition}
	Let $f:\CC\rightarrow\CC$, where $f(x+iy)=u(x,y)+iv(x,y)$, be a function such that $\partial u/\partial x$, $\partial u/\partial y$, $\partial v/\partial x$, and $\partial v/\partial y$ exist and are continuous on some disk $D\subseteq\CC$ with a nonzero radius.
	If $f$ satisfies the \textit{Cauchy-Riemann equations}
	\[
		\frac{\partial u}{\partial x} = \frac{\partial v}{\partial y} \text{ and } \frac{\partial u}{\partial y} = -\frac{\partial v}{\partial x}
	\]
	then $f$ is said to be \textit{analytic on $D$}.
	Furthermore, the largest subset of $\CC$ on which $f$ is analytic is called $f$'s \textit{domain of analyticity}.
	If the domain of analyticity is $\CC$, then $f$ is said to be \textit{entire}.
\end{definition}
Since any analytic function is infinitely differentiable, its Taylor expansion exists as well.
Furthermore, the Taylor series converges on any disk in the domain of analyticity.
\begin{thm}
	Let $f$ be analytic on a disk $D(z_0,r)$, then $f$'s Taylor series about $z_0$ converges for all $z\in D$.
\end{thm}

\section{Solution}

\begin{problem*}
	Show that for any $z,w\in\CC$,
	\[
		\sin(z+w)=\cos z\sin w+\sin z\cos w \text{ and } \cos(z+w)=\cos z\cos w - \sin z\sin w.
	\]
\end{problem*}

Our solution will give $\sin z$ and $\cos z$ in terms of $e^z$.
From there, we will show the sum formulae.
However, before we can get to either of those steps, we need to show that $e^z$ is entire.

\subsection{Analyticity of $e^{z}$}

\begin{proposition}
	$e^z$ is entire.
\end{proposition}
\begin{proof}
	Let $z=x+iy$.
	We want to find $\RR$-valued functions $u,v$ such that $e^{x+iy}=u(x,y)+iv(x,y)$.
	\begin{align*}
		e^z &= e^{x+iy}\\
		&= e^x e^{iy}\\
		&= e^x \cos y + ie^x\sin y\\
		&= u(x,y) +iv(x,y)
	\end{align*}
	Taking partial derivatives yields the following.
	\[
		\begin{array}{rlrl}
			\partial u/\partial x =& e^x \cos y  \hspace{1cm}&  \partial v/\partial y =& e^x \cos y\\
			\partial u/\partial y =& -e^x \sin y \hspace{1cm}& -\partial v/\partial x =& -e^x \sin y
		\end{array}
	\]
	Therefore $e^z$ satisfies the Cauchy-Riemann equations.
	Furthermore, since the equations hold for all $z\in\CC$, $e^z$ is entire.
\end{proof}

\subsection{Complex Trigonometric Functions in Terms of the Complex Exponential}

Our next step is to use the Taylor expansion of $e^z$ to get the Taylor expansion of $\cos z$ and $\sin z$.

\begin{proposition}
	Let $z\in\CC$.
	Then
	\[
		\cos z = \frac{ e^{iz} + e^{-iz} }{2} \text{ and } \sin z = \frac{ e^{iz} - e^{-iz} }{2i}.
	\]
\end{proposition}
\begin{proof}
	Consider $\paren{e^{iz}+e^{-iz}}/2$.
	Since $e^z$ is entire, its Taylor series converges everywhere in $\CC$.
	\begin{align*}
		\frac{1}{2}\paren{e^{iz}+e^{-iz}}
		&= \frac{1}{2}\paren{\sum_{n=0}^\infty \frac{1}{n!} i^n z^n + \frac{(-1)^n}{n!} i^n z^n}\\
		&= \frac{1}{2}\paren{\sum_{n=0}^\infty \frac{1+(-1)^n}{n!} i^n z^n}\\
		&= \frac{1}{2}\paren{\sum_{n=0}^\infty \frac{1+(-1)^{2n}}{(2n)!}i^{2n}z^{2n} + \frac{1+(-1)^{2n+1}}{(2n+1)!}i^{2n+1}z^{2n+1}}\\
		&= \frac{1}{2}\paren{\sum_{n=0}^\infty \frac{2}{(2n)!}(i^2)^nz^{2n}}\\
		&= \sum_{n=0}^\infty \frac{(-1)^n}{(2n!)} z^{2n}\\
		&= \cos z
	\end{align*}
	Consider $\paren{e^{iz}-e^{-iz}}/(2i)$.
	\begin{align*}
		\frac{1}{2i}\paren{e^{iz}-e^{-iz}}
		&= \frac{1}{2i}\paren{\sum_{n=0}^\infty \frac{1}{n!} i^n z^n - \frac{(-1)^n}{n!} i^n z^n}\\
		&= \frac{1}{2i}\paren{\sum_{n=0}^\infty \frac{1-(-1)^n}{n!} i^n z^n}\\
		&= \frac{1}{2i}\paren{\sum_{n=0}^\infty \frac{1-(-1)^{2n}}{(2n)!}i^{2n}z^{2n} + \frac{1-(-1)^{2n+1}}{(2n+1)!}i^{2n+1}z^{2n+1}}\\
		&= \frac{1}{2i}\paren{\sum_{n=0}^\infty \frac{2i}{(2n+1)!}(i^2)^nz^{2n+1}}\\
		&= \sum_{n=0}^\infty \frac{(-1)^n}{(2n+1)!} z^{2n+1}\\
		&= \sin z
	\end{align*}
\end{proof}

\subsection{Trigonometric Sum Identities}

We will now use our new identities to show the sum formulae, and get the difference formulae as corollaries.

\begin{thm}
	Let $z,w\in\CC$.
	Then $\sin(z+w)=\cos z\sin w+\sin z\cos w$.
\end{thm}
\begin{proof}
	Consider $\cos z\sin w+\sin z\cos w$.
	\begin{align*}
		\cos z\sin w+\sin z\cos w &=
		\paren{\frac{e^{iz}+e^{-iz}}{2}}\paren{\frac{e^{iw}-e^{-iw}}{2i}}+\paren{\frac{e^{iz}-e^{-iz}}{2i}}\paren{\frac{e^{iw}+e^{-iw}}{2}}\\
		&= \frac{e^{i(z+w)} +e^{i(w-z)} - e^{i(z-w)} - e^{-i(z+w)}}{4i} + \frac{e^{i(z+w)} -e^{i(w-z)} + e^{i(z-w)} - e^{-i(z+w)}}{4i}\\
		&= \frac{2e^{i(z+w)}-2e^{-i(z+w)}}{4i}\\
		&= \frac{e^{i(z+w)}-e^{-i(z+w)}}{2i}\\
		&= \sin(z+w)
	\end{align*}
\end{proof}

\begin{thm}
	Let $z,w\in\CC$.
	Then $\cos(z+w)=\cos z\cos w - \sin z\sin w$.
\end{thm}
\begin{proof}
	Consider $\cos z\cos w - \sin z\sin w$.
	\begin{align*}
		\cos z\cos w - \sin z\sin w &=
		\paren{\frac{e^{iz}+e^{-iz}}{2}}\paren{\frac{e^{iw}+e^{-iw}}{2}} - \paren{\frac{e^{iz}-e^{-iz}}{2i}}\paren{\frac{e^{iw}-e^{-iw}}{2i}}\\
		&= \frac{e^{i(z+w)} + e^{i(w-z)} + e^{i(z-w)} + e^{-i(z+w)}}{4} + \frac{e^{i(z+w)} -e^{i(w-z)} - e^{i(z-w)} + e^{-i(z+w)}}{4}\\
		&= \frac{e^{i(z+w)}+e^{-i(z+w)}}{2}\\
		&= \cos(z+w)
	\end{align*}
\end{proof}

\begin{corollary}
	Let $z,w\in\CC$.
	Then
	\[
		\sin(z-w)=\sin z\cos w - \cos z\sin w \text{ and }\cos(z-w)=\cos z\cos w + \sin z\sin w.
	\]
\end{corollary}
\begin{proof}
	Since $\sin w$ is an odd function, $\sin(-w)=-\sin w$.
	Since $\cos w$ is an even function, $\cos (-w)=\cos w$.
	These facts combined with the sum formulas for $\cos(z+(-w))$ and $\sin(z+(-w))$ yield the difference formulae.
\end{proof}

\printbibliography

\end{document}
