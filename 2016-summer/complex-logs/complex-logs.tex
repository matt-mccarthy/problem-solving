\documentclass[notitlepage]{problem-solving}

\title{The Mystery of $1^\pi$\\ Wrestling with Complex Exponentiation}
\author{Matt McCarthy}
\date{June 2016}

\addbibresource{complex-logs.bib}

\nocite{*}

\begin{document}

\maketitle

\begin{problem*}
	Find what step is wrong in the following statements.
	\begin{align*}
		1 &= 1^\pi\\
		&= \paren{e^{2i\pi}}^\pi\\
		&=e^{2i\pi^2}\\
		&=\cos 2\pi^2 + i\sin 2\pi^2\\
		&\approx 0.6296 +0.7768 i
	\end{align*}
\end{problem*}

\section{Background}

\subsection{Complex Logarithms}

In $\RR$ we can define logarithms in various ways such as the inverse of the exponential or as
\[
	\ln x = \int \frac{1}{x} dx.
\]
However, in $\CC$ the original definition (inverse of the exponential) fails since
\[
	e^{i\theta} = e^{i\theta + 2ki\pi}
\]
where $k\in\ZZ$.
Essentially, the exponential is a bijection between $\RR$ and $(0,\infty)$ and is thus invertible.
When we move to $\CC$, the exponential function loses its injectivity since multiple domain elements get mapped to a single element in the range.
However, defining logarithms as the antiderivative of $1/x$ still works.
\begin{definition}
	Let $z\in\CC$.
	Then $\log z$ is defined as
	\[
		\log z := \ln |z| +i\arg z.
	\]
	If $\arg z\in(-\pi,\pi]$ we say it is the \textit{principal logarithm} of $z$.
\end{definition}
\begin{corollary}
	\[
		\frac{d}{dz} \log z = \frac{1}{z}
	\]
\end{corollary}
As we can see by the definition, the complex logarithm is \textit{multivalued}, thus if we want it to be well-defined we need to restrict the argument of the input to an open interval of length $2\pi$.

\subsection{Exponentiation with a Complex Base}

Because of these problems with the logarithm, exponentiation becomes less straightforward.
In the complex world, exponentiation is defined as follows.
\begin{definition}
	Let $z,w\in\CC$ with $w\neq 0$.
	Then
	\[
		w^z = e^{z\log w}.
	\]
\end{definition}
However, this definition is equivalent to
\[
	w^z = e^{z(\ln|w| + i\arg w)}.
\]
Thus, due to the properties of the complex logarithm, exponentiation in the complex plane is also multivalued.

When exponentiating, the following are true.
\begin{enumerate}
	\item $z^w$ is single-valued iff $w\in\ZZ$.
	\item $z^{m/n}$ has $n$ distinct values for $m,n\in\ZZ$ with $n>0$ and $\gcd(m,n)=1$.
	\item $z^x$ has infinitely many values if $x$ is irrational.
\end{enumerate}

\section{Solution}

\begin{problem*}
	Find what step is wrong in the following statements.
	\begin{align*}
		1 &= 1^\pi\\
		&= \paren{e^{2i\pi}}^\pi\\
		&=e^{2i\pi^2}\\
		&=\cos 2\pi^2 + i\sin 2\pi^2\\
		&\approx 0.6296 +0.7768 i
	\end{align*}
\end{problem*}

The problem is
\[
	1^\pi = 1.
\]
By definition of complex exponentiation we have
\[
	1^\pi = e^{\pi\log 1}.
\]
However
\[
	\log 1 = \ln |1| + i\arg 1 = 0 + 2ki\pi
\]
for $k\in\ZZ$.
Thus,
\[
	1^{\pi} = e^{2ki\pi^2}.
\]
Since $\pi$ is irrational, there are infinitely many values that $1^\pi$ can take.
The only value of $k$ that yields an answer of 1 is $k=0$.

\printbibliography

\end{document}
