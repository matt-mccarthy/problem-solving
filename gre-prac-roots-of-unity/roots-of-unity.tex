\documentclass[notitlepage]{problem-solving}

\author{Matt McCarthy}
\date{June 2016}
\title{Roots of Unity}

\begin{document}

\maketitle

\begin{problem*}
	Find the roots of the equation
	\[
		z^5-1=0.
	\]
\end{problem*}

\section{Background}

The complex numbers, denoted as $\CC$ are defined as follows.
\begin{definition}
	The set of \textit{complex numbers} is the following two-dimensional vector space over the real numbers.
	\[
		\CC := \set{a+bi\, :\, a,b\in\RR}
	\]
\end{definition}
Furthermore, all $z\in\CC$ have the form
\[
	r(\cos\theta+i\sin\theta)
\]
where $r\in\RR$ with $r\geq 0$ and $\theta\in\RR$.
Additionally, we have Euler's formula which will allow us to write the polar form of a complex number more concisely.
\begin{thm}
	For all $\theta\in\RR$,
	\[
		e^{i\theta} = \cos\theta+i\sin\theta.
	\]
\end{thm}
Thus, any complex number, $z$, can be written as
\[
	z=re^{i\theta}.
\]
Moreover, for any $z=a+bi=re^{i\theta}\in\CC$, the \textit{conjugate} of $z$, denoted $\bar{z}$, is defined as
\[
	\bar{z}:=a-bi=re^{-i\theta}
\]
and the \textit{modulus} of a $z$ is defined as
\[
	|z| := \sqrt{a^2+b^2} = r.
\]
Thus,
\[
	|e^{i\theta}| = 1.
\]
Lastly, an \textit{$n$th root of unity} is a solution to the equation
\[
	z^n = 1.
\]

\section{Solution}

\begin{problem*}
	Find the roots of the equation
	\[
		z^5-1=0.
	\]
\end{problem*}

We want to find the roots of the equation,
\[
	z^5 - 1= 0.
\]
Equivalently, we will find the 5th roots of unity, or the solutions to
\[
	z^5 = 1.
\]
To begin, we know that $z=re^{i\theta}$ for some $r\geq 0$ and $\theta\in\RR$.
Furthermore, we know that $1=e^{2ki\pi}$ for all $k\in\ZZ$.
Thus,
\[
	r^5e^{5i\theta} = e^{2ki\pi}
\]
for all $k\in\ZZ$.
Since $|e^{i\phi}| = 1$ for all $\phi\in\RR$, we know that $r^5 = 1$.
Thus, $r=1$ because $r$ is a positive real.
Thus, we are left with
\[
	e^{5i\theta}=e^{2ki\pi}.
\]
Ergo,
\[
	\theta = 2ki\pi/5
\]
for all $k\in\ZZ$.
However, this is an infinite solution set and there are only 5 \textit{distinct} solutions.
To find these distinct solutions, we use the fact that
\[
	e^{i\theta} = e^{i(\theta+2k\pi)}
\]
for any $k\in\ZZ$.
Thus, our distinct values for $\theta$ are as follows.
\[
	\theta\in\set{0, 2\pi/5, 4\pi/5, 6\pi/5, 8\pi/5}
\]
When $\theta =10\pi/5 = 2\pi$, we get the same result as when $\theta=0$ since $2\pi\equiv 0\mod{2\pi}$.
Therefore,
\[
	z\in\set{1,e^{2\pi/5}, e^{4\pi/5}, e^{6\pi/5}, e^{8\pi/5}}.
\]

\end{document}
