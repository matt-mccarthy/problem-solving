\documentclass[notitlepage]{simple}

\title{Algebraic Properties of the Gaussian Integers}
\author{Matt McCarthy}
\date{April 2016}

\begin{document}

\maketitle

\begin{thm*}
	The Gaussian Integers, denoted $\ZZ(i)$, form a Euclidean domain.
\end{thm*}

\section{Background}

Before we can talk about Euclidean domains, we first need to introduce the definition of a ring.

\begin{definition}[Ring]
	Let $R$ be a nonempty set, and let $+:R^2\rightarrow R$ and $\cdot:R^2\rightarrow R$ be binary operations on $R$.
	Then we say $R$ is a \textit{ring} if all of the following hold.
	\begin{enumerate}
		\item The structure $(R,+)$ is an abelian group.
		\item For any $a,b,c\in R$, $a(bc)=(ab)c$ (Multiplicative Associativity).
		\item For any $a,b,c\in R$, $a(b+c)=ab+ac$ (Left Distributivity).
		\item For any $a,b,c\in R$, $(a+b)c=ac+bc$ (Right Distributivity).
	\end{enumerate}
\end{definition}

\end{document}
